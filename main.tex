\documentclass[bibliography=totocnumbered]{scrartcl}

\usepackage[utf8]{inputenc} %UTF8 without BOM!
\usepackage[T1]{fontenc}
\usepackage[british]{babel}

\usepackage{amssymb,amsmath,mathtools, booktabs}

\usepackage[style=alphabetic,backend=biber,hyperref=true,sortcites=true,maxbibnames=6,minbibnames=3,maxcitenames=1,mincitenames=1]{biblatex}
\usepackage[strict=true]{csquotes}
\addbibresource{main.bib}

\usepackage[linktocpage=true]{hyperref} %linktocpage=true: Zahlen im Inhaltsverzeichnis verlinken, antatt Text. %Um Links nicht farbig zu machen benutze: pdfborder={0 0 0} oder colorlinks=false,allbordercolors=white
\usepackage[anythingbreaks]{breakurl}

\title{Exponential Integral $E_1\left(x\right)$}
\author{André Wählisch\\\href{mailto:andre.waehlisch@physik.tu-berlin.de}{andre.waehlisch@physik.tu-berlin.de}}
\date{Last update: \today}

\newcommand{\assume}[1][\text{MISSING PARAMETER}]{,\qquad\left(#1\right)}

\begin{document}
	\maketitle
	\tableofcontents
	\clearpage
	
	\section{Pre-stuff}
	
	\subsection{Informational}
	
	\begin{itemize}
		\item Github: \url{https://github.com/AndreWaehlisch/Exponential-Integral}
		\item Some links about the exponential integral:
		\begin{itemize}
			\item wolfram.com:
			\begin{itemize}
				\item $E_1$ and $Ei$:
				\begin{itemize}
					\item \url{http://mathworld.wolfram.com/ExponentialIntegral.html}
					\item \url{http://functions.wolfram.com/GammaBetaErf/ExpIntegralEi/}
					\item \url{http://www.wolframalpha.com/input/?i=ExpIntegralE[1,x]}
					\item \url{http://www.wolframalpha.com/input/?i=ExpIntegralEi[x]}
				\end{itemize}
				\item $E_n$:
				\begin{itemize}
					\item \url{http://mathworld.wolfram.com/En-Function.html}
					\item \url{http://functions.wolfram.com/GammaBetaErf/ExpIntegralE/}
				\end{itemize}
			\end{itemize}
		\end{itemize}
	\end{itemize}
	
	\subsection{Mathematical notes}
	
	\begin{enumerate}
		\item All variables and parameters are considered real, if not stated otherwise.
		\item The notation $\left[f\left(x\right)\right]_{x=a}^{b}$ means $f\left(b\right)-\left(a\right)$. In the same sense, if for example $b=\infty$, interpret as $\left[f\left(x\right)\right]_{x=a}^{\infty}=\lim_{x\rightarrow{}\infty}\left(f\left(x\right)\right)-f\left(a\right)$.
	\end{enumerate}
	
	\subsection{Definitions}
	
	\subsubsection[Definition of Exponential Integral E1(x)]{Definition of Exponential Integral $E_1\left(x\right)$}
	
	\begin{gather}
		E_1\left(x\right)=\int_{x}^{\infty}\frac{e^{-w}}{w}dw\assume[x>0]\label{eq: E1}
	\end{gather}
	
	\subsubsection[Definition of D(x)]{Definition of $D\left(x\right)$}
	\label{subsubsec: D}
	
	\begin{gather}
		D\left(x\right)=e^xE_1\left(x\right)\assume[x>0]\label{eq: D}
	\end{gather}
	
	\subsubsection[Definition of Y(a,b,c,d)]{Definition of $Y\left(a, b, c, d\right)$}
	
	\begin{gather}
		Y\left(a, b, c, d\right)=\int_{0}^{d}e^{ax}E_1\left[b\left(x+c\right)\right]dx\assume[d>0,b>0,c>0]\label{eq: Y}
	\end{gather}
	
	\clearpage
	
	\section[Exponential Integral E1(x)]{Exponential Integral $E_1\left(x\right)$}
	
	\subsection{Integral Representations}
	
	\subsubsection[A integral representation not involving x in the integral limits]{$E_1\left(x\right)=\int_{1}^{\infty}\frac{e^{-xt}}{t}dt$, source is eq. (2b) of \cite{boer1990calc}}
	\label{subsubsec: def01}
	
	Use substitution $w=xt$:
	
	\begin{align}
		E_1\left(x\right)&=\int_{x}^{\infty}\frac{e^{-w}}{w}dw\assume[x>0]\\
		&=\int_{1}^{\infty}\frac{e^{-xt}}{t}dt\label{eq: integralrepresentation1}
	\end{align}
	
	\subsubsection[A integral representation not involving exp(x) in the integrand]{$E_1\left(x\right)=e^{-x}\int_{0}^{1}\frac{1}{x-\ln{t}}dt$, source is eq. (4) in sec. 3.3 of \cite{geller1969table}}
	\label{subsubsec: intrep01}
	\begin{align}
		E_1\left(x\right)&=\int_{x}^{\infty}\frac{e^{-w}}{w}dw\assume[x>0]\\
		&=e^{-x}\int_{x}^{\infty}\frac{e^{x-w}}{w}dw
	\end{align}
	
	First use substitution $\left(x-w\right)=-y$, then use substitution $y=-\ln{t}$:
	
	\begin{align}
		e^{-x}\int_{x}^{\infty}\frac{e^{x-w}}{w}dw&=e^{-x}\int_{0}^{\infty}\frac{e^{-y}}{x+y}dy\\
		&=e^{-x}\int_{1}^{0}\frac{t}{x-\ln{t}}\left(\frac{-1}{t}\right)dt\\
		&=e^{-x}\int_{0}^{1}\frac{1}{x-\ln{t}}dt
	\end{align}
	
	\subsubsection[A integral representation]{$E_1\left(x\right)=e^{-x}\int_{0}^{\infty}\frac{e^{-xt}}{t+1}dt$}
	\label{subsubsec: intrep02}
	
	Use substitution $\frac{w}{x}-1=t$:
	
	\begin{align}
		E_1\left(x\right)&=\int_{x}^{\infty}\frac{e^{-w}}{w}dw\\
		&=\int_{0}^{\infty}\frac{e^{-x\left(t+1\right)}}{x\left(t+1\right)}\left(xdt\right)\\
		&=e^{-x}\int_{0}^{\infty}\frac{e^{-xt}}{t+1}dt
	\end{align}

	\subsubsection[A integral representation involving integration over an angle]{$E_1\left(x\right)=\int_{0}^{\frac{\pi}{2}}\tan{\left(\alpha\right)}e^{-\frac{x}{\cos{\alpha}}}d\alpha$, this is eq. (2a) from \cite{boer1990calc}}

	Starting with the representation of equation \eqref{eq: integralrepresentation1} and using the substitution $t=\frac{1}{\cos\alpha}$ we immediately arrive at the result:
	
	\begin{align}
		E_1\left(x\right)&=\int_{1}^{\infty}\frac{e^{-xt}}{t}dt\\
		&=\int_{0}^{\frac{\pi}{2}}\cos{\left(\alpha\right)}e^{-\frac{x}{\cos\alpha}}\frac{\tan\alpha}{\cos\alpha}d\alpha\\
		&=\int_{0}^{\frac{\pi}{2}}\tan{\left(\alpha\right)}e^{-\frac{x}{\cos{\alpha}}}d\alpha
	\end{align}

	\subsection{Special Values}
	
	\subsubsection[A derivative of E1]{$\frac{d}{dx}E_1\left[b\left(x+c\right)\right]=-\frac{e^{-b\left(x+c\right)}}{x+c}$}
	\label{subsubsec: specval_deriv}
	
	Start with integral representation of $E_1\left(x\right)$ found in subsection \ref{subsubsec: def01}:
	
	\begin{align}
		\frac{d}{dx}E_1\left[b\left(x+c\right)\right]&=\frac{d}{dx}\int_{1}^{\infty}\frac{e^{-b\left(x+c\right)t}}{t}dt\\
		&=\int_{1}^{\infty}\left(-bt\right)\frac{e^{-b\left(x+c\right)t}}{t}dt\\
		&=-b\int_{1}^{\infty}e^{-b\left(x+c\right)t}dt\\
		&=\frac{-b}{-b\left(x+c\right)}\left[e^{-b\left(x+c\right)t}\right]^{\infty}_{t=1}\\
		&=-\frac{e^{-b\left(x+c\right)}}{x+c}\qquad\left(b\left(x+c\right)>0\right)
	\end{align}

	\subsection[Integrals involving E1(x)]{Integrals involving $E_1\left(x\right)$}
	
	\begin{table}[h]
		\centering
		\begin{tabular}{cccc}
			\toprule
			Integral & $Y$-representation & Location & Location in \cite{boer1990calc}\\
			\midrule
			$\int_{0}^{d}e^{ax}E_1\left[b\left(x+c\right)\right]dx$ & $Y\left(a,b,c,d\right)$ & \eqref{eq: generalIntegralOfE1}, p. \pageref{eq: generalIntegralOfE1} & eq. (A2a)\\
			$\int_{0}^{d}e^{ax}E_1\left[bd\right]dx$ & $Y\left(a,b,0,d\right)$ & \eqref{eq: generalIntegralOfE4}, p. \pageref{eq: generalIntegralOfE4} & eq. (A2b)\\
			$\int_{0}^{\infty}e^{ax}E_1\left[b\left(x+c\right)\right]dx$ & $Y\left(a,b,c,\infty\right)$ & \eqref{eq: generalIntegralOfE2}, p. \pageref{eq: generalIntegralOfE2} & eq. (A2c)\\
			$\int_{0}^{\infty}e^{ax}E_1\left[bx\right]dx$ & $Y\left(a,b,0,\infty\right)$ & \eqref{eq: generalIntegralOfE3}, p. \pageref{eq: generalIntegralOfE3} & eq. (A2d)\\
			\bottomrule
		\end{tabular}
		\caption{Table of integrals encountered in \cite{boer1990calc}}
		\label{tab: tableOfIntegrals}
	\end{table}
	
	\subsubsection[A definite integral having two exponential integrals as result]{$\int_{a}^{b}\frac{e^{-x}}{x}dx=E_1\left(a\right)-E_1\left(b\right)$}
	\label{subsubsec: definiteIntegral}
	
	Start with the negative derivative from subsection \ref{subsubsec: specval_deriv}, while setting $b=1$ and $c=0$:
	
	\begin{gather}
		-\frac{d}{dx}E_1\left(x\right)=\frac{e^{-x}}{x}\label{eq: derivE1(x)}
	\end{gather}
	
	Now, just apply the \emph{fundamental theorem of calculus} (in German: \emph{Hauptsatz der Differential- und Integralrechnung}): For certain conditions to $f\left(x\right), a, b$ and if $\frac{d}{dx}F\left(x\right)=f\left(x\right)$, then $\int_{a}^{b}f\left(x\right)dx=F\left(b\right)-F\left(a\right)$. Applying this to equation \eqref{eq: derivE1(x)} leads to the desired result. We assume $a\neq{}b$, so that the integral is not trivially zero. Also required due to the infinite discontinuity of the integrand at $x=0$ is that $a$ and $b$ are \emph{both} positive or \emph{both} negative, furthermore both must be non-zero:

	\begin{gather}
		\int_{a}^{b}\frac{e^{-x}}{x}dx=E_1\left(a\right)-E_1\left(b\right)\qquad\left(a\neq{}b, ab>0\right)
	\end{gather}
	
	\subsubsection[A integral of E1, involving the exponential function]{$\int_{0}^{d}e^{ax}E_1\left[b\left(x+c\right)\right]dx=\frac{e^{-bc}}{a}\left(e^{\left(a-b\right)d}\left\{D\left[b\left(c+d\right)\right]-D\left[\left(-a+b\right)\left(c+d\right)\right]\right\}-D\left[bc\right]+D\left[\left(-a+b\right)c\right]\right)$}
	\label{subsubsec: generalIntegralOfE1}
	
	This is the general case of $Y\left(a, b, c, d\right)=\int_{0}^{d}e^{ax}E_1\left[b\left(x+c\right)\right]dx$, found in \cite{boer1990calc}. Special cases, e.g. $c=0$ are considered in the following subsections. Now, first integrate by parts, using the derivative $\frac{d}{dx}E_1\left[b\left(x+c\right)\right]=-\frac{e^{-b\left(x+c\right)}}{x+c}$ from subsection \ref{subsubsec: specval_deriv}:
	
	\begin{align}
		\int_{0}^{d}e^{ax}E_1\left[b\left(x+c\right)\right]&=\left\{\frac{e^{ax}}{a}E_1\left[b\left(x+c\right)\right]\right\}_{x=0}^d-\int_{0}^{d}\frac{e^{ax}}{a}\frac{d}{dx}E_1\left[b\left(x+c\right)\right]dx\\
		&=\frac{e^{ad}}{a}E_1\left[b\left(d+c\right)\right]-\frac{E_1\left[bc\right]}{a}-\int_{0}^{d}\frac{e^{ax}}{a}\left\{-\frac{e^{-b\left(x+c\right)}}{x+c}\right\}dx\\
		&=\frac{e^{ad}}{a}E_1\left[b\left(d+c\right)\right]-\frac{E_1\left[bc\right]}{a}+\frac{e^{-bc}}{a}\int_{0}^{d}\frac{e^{-x\left(b-a\right)}}{x+c}dx\label{eq: stopEqOfFirstIntegral}
	\end{align}
	
	To simplify the integral in the last summand, first use the abbreviation $\phi=b-a$ and the substitution $x+c=y$, then use the substitution $\left(b-a\right)y=\phi{}y=t$. In the final step, to get to the intermediate result of equation \eqref{eq: intermediateResult}, use the integral $\int_{a}^{b}\frac{e^{-x}}{x}dx=E_1\left(a\right)-E_1\left(b\right)$, found in subsection  \ref{subsubsec: definiteIntegral}:
	
	\begin{align}
		\int_{0}^{d}\frac{e^{-x\left(b-a\right)}}{x+c}dx&=\int_{0}^{d}\frac{e^{-x\phi}}{x+c}dx\\
		&=\int_{c}^{d+c}\frac{e^{-\left(y-c\right)\phi}}{y}dy\\
		&=e^{c\phi}\int_{c}^{d+c}\frac{e^{-y\phi}}{y}dy\\
		&=e^{c\phi}\int_{\phi{}c}^{\phi\left(d+c\right)}\frac{e^{-t}}{t\phi^{-1}}\phi^{-1}dt\\
		&=e^{c\phi}\left\{E_1\left[\phi{}c\right]-E_1\left[\phi\left(d+c\right)\right]\right\}\label{eq: intermediateResult}
	\end{align}
	
	To arrive at the desired result we just put the intermediate step \eqref{eq: intermediateResult} into equation \eqref{eq: stopEqOfFirstIntegral}, and restore the abbreviation $\phi=b-a$. Then rearrange the terms to get to equation (A2a) from the appendix of \cite{boer1990calc}. In the last steps use the definition of $D\left(x\right)$ from eq. \eqref{eq: D}, which can be found in subsection \ref{subsubsec: D} on page \pageref{subsubsec: D}:
	
	\begin{align}
			\begin{split}
				\int_{0}^{d}e^{ax}E_1\left[b\left(x+c\right)\right]=&\frac{e^{ad}}{a}E_1\left[b\left(d+c\right)\right]-\frac{E_1\left[bc\right]}{a}\\
				&+\frac{e^{-bc}e^{c\left(b-a\right)}}{a}\left\{E_1\left[\left(b-a\right)c\right]-E_1\left[\left(b-a\right)\left(d+c\right)\right]\right\}
			\end{split}\\
			\begin{split}
				=&\frac{1}{a}\left\{e^{ad}E_1\left[b\left(c+d\right)\right]-e^{-ac}E_1\left[\left(-a+b\right)\left(c+d\right)\right]\right. \\
				&\left. {}-E_1\left[bc\right]+e^{-ac}E_1\left[\left(-a+b\right)c\right]\vphantom{e^{ad}}\right\}
			\end{split}\\
			\begin{split}
				=&\frac{e^{-bc}}{a}\left(e^{\left(a-b\right)d}\left\{D\left[b\left(c+d\right)\right]-D\left[\left(-a+b\right)\left(c+d\right)\right]\right\}\right.\\
				&\left. {}-D\left[bc\right]+D\left[\left(-a+b\right)c\right]\right)\label{eq: generalIntegralOfE1}
			\end{split}
	\end{align}
	
	Note that -- as always when using the definition of $E_1$ from equation \eqref{eq: E1}  -- the arguments of the exponential integrals $E_1\left(x\right)$ and $D\left(x\right)$ must be positive:
	
	\begin{subequations}
	\begin{align}
		b\left(c+d\right)&>0\\
		\left(-a+b\right)\left(c+d\right)&>0\\
		bc&>0\label{eq: conditionForC1}\\
		\left(-a+b\right)c&>0\label{eq: conditionForC2}
	\end{align}
	\end{subequations}
	
	We can see, that for the interesting special case $c=0$, the result is not defined: It violates the conditions \eqref{eq: conditionForC1} and \eqref{eq: conditionForC2}.
	
	\subsubsection[A integral of E1, involving the exponential function]{$\int_{0}^{\infty}e^{ax}E_1\left[b\left(x+c\right)\right]dx=\frac{1}{a}e^{-bc}\left\{-D\left[bc\right]+D\left[c\left(-a+b\right)\right]\right\}$}
	\label{subsubsec: generalIntegralOfE2}
	
	Start with the integral representation $E_1\left(x\right)=\int_{1}^{\infty}\frac{e^{-xt}}{t}dt$ from subsection \ref{subsubsec: def01}:
	
	\begin{align}
		\int_{0}^{\infty}e^{ax}E_1\left[b\left(x+c\right)\right]dx&=\int_{0}^{\infty}e^{ax}\int_{1}^{\infty}\frac{e^{-b\left(x+c\right)t}}{t}dtx\\
		&=\int_{1}^{\infty}dt\frac{e^{-bct}}{t}\int_{0}^{\infty}dxe^{x\left(a-bt\right)}\\
		&=\int_{1}^{\infty}dt\frac{e^{-bct}}{t}\left(-\frac{1}{a-bt}\right)\assume[a-bt<0]\label{eq: restrictionOfGeneralIntegralOfE2}\\
		&=-\int_{1}^{\infty}dt\frac{e^{-bct}}{a}\frac{a-bt+bt}{t\left(a-bt\right)}\\
		&=-\int_{1}^{\infty}dt\frac{e^{-bct}}{a}\left(\frac{1}{t}+\frac{b}{a-bt}\right)
	\end{align}
	
	The first term in the last equation we recognize as $E_1\left(bc\right)$. For the second summand we first apply the substitution $-\left(a-bt\right)=y$ and then $\frac{y}{b-a}=x$ to also get an exponential integral:
	
	\begin{align}
		\int_{0}^{\infty}e^{ax}E_1\left[b\left(x+c\right)\right]dx&=-\frac{1}{a}\left\{E_1\left[bc\right]-\int_{b-a}^{\infty}\frac{be^{-bc\frac{a+y}{b}}}{y}\frac{dy}{b}\right\}\\
		&=-\frac{1}{a}\left\{E_1\left[bc\right]-e^{-ac}\int_{b-a}^{\infty}\frac{e^{-cy}}{y}dy\right\}\\
		&=-\frac{1}{a}\left\{E_1\left[bc\right]-e^{-ac}\int_{1}^{\infty}\frac{e^{-c\left(b-a\right)x}}{\left(b-a\right)x}\left(b-a\right)dx\right\}\\
		&=-\frac{1}{a}\left\{E_1\left[bc\right]-e^{-ac}E_1\left[c\left(-a+b\right)\right]\right\}\\
		&=\frac{1}{a}e^{-bc}\left\{-D\left[bc\right]+D\left[c\left(-a+b\right)\right]\right\}\label{eq: generalIntegralOfE2}
	\end{align}
	
	In the end we arrive at the equation (A2c) found in the appendix of \cite{boer1990calc}. We have to comply to the following restrictions:
	
	\begin{subequations}
		\begin{gather}
			a-bt<0\overset{\eqref{eq: restrictionOfGeneralIntegralOfE2}}{\Rightarrow}a-b<0\Rightarrow{}a<b\\
			bc>0
		\end{gather}
	\end{subequations}
	
	\subsubsection[A integral of E1, involving the exponential function]{$\int_{0}^{\infty}e^{ax}E_1\left(bx\right)dx=-\frac{1}{a}\ln\left(1-\frac{a}{b}\right)$}
	\label{subsubsec: generalIntegralOfE3}
	
	Start with the integral representation $E_1\left(bx\right)=e^{-bx}\int_{0}^{\infty}\frac{e^{-bxt}}{t+1}dt$ from subsection \ref{subsubsec: intrep02}, which is only valid for $bx>0$. Then multiply the equation with $e^{ax}dx$, while integrating both sides from $x=0$ to infinity. After that just change the integration order:
	
	\begin{align}
		\int_{0}^{\infty}e^{ax}E_1\left(bx\right)dx&=\int_{0}^{\infty}e^{ax}\left[e^{-bx}\int_{0}^{\infty}\frac{e^{-bxt}}{t+1}dt\right]dx\assume[b>0]\\
		&=\int_{0}^{\infty}\frac{dt}{1+t}\int_{0}^{\infty}e^{-\left(-a+b+bt\right)x}dx\\
		&=\int_{0}^{\infty}\frac{1}{1+t}\frac{1}{\left(-a+b+bt\right)}dt\assume[-a+b+bt>0]
	\end{align}
	
	We can see the requirement $-a+b+bt>0$ yields $-a+b>0$ for the integration limit $t=0$, or equivalently $a<b$. On this result we want to use the substitution $\frac{1}{1+t}=z$. Then, in the next step, substitute $b-az=x$:
	
	\begin{align}
		\int_{0}^{\infty}e^{ax}E_1\left(bx\right)dx&=\int_{1}^{0}z\frac{1}{\left(-a+\frac{b}{z}\right)}\left(-\frac{1}{z^2}dz\right)\\
		&=\int_{0}^{1}\frac{1}{\left(b-az\right)}dz\\
		&=\int_{b}^{b-a}\frac{1}{x}\left(-\frac{1}{a}dx\right)\\
		&=-\frac{1}{a}\left[\ln{x}\right]^{b-a}_{x=b}\\
		&=-\frac{1}{a}\left[\ln{\left(b-a\right)}-\ln{\left(b\right)}\right]\\
		&=-\frac{1}{a}\ln{\left(1-\frac{a}{b}\right)}\assume[a<b, 0<b]\label{eq: generalIntegralOfE3}
	\end{align}
	
	This end result is equation (A2d) from \cite{boer1990calc}. In terms of the $Y$-function this is $Y\left(a,b,0,\infty\right)$. For a more general derivation based on that of \autocite[73\psq]{schloemilch} see section \ref{appsec: moreGeneralDerivation1} in the appendix. The equation can also be found as equation (R10) in \cite{sherman}.
	
	\subsubsection[A integral of E1, involving the exponential function]{$\int_{0}^{d}e^{ax}E_1\left(bx\right)dx=\frac{1}{a}\left(e^{\left(a-b\right)d}\left\{D\left[bd\right]-D\left[\left(-a+b\right)d\right]\right\}-\ln{\left[1-\frac{a}{b}\right]}\right)$}
	
	To start we plug in the integral representation $E_1\left(x\right)=\int_{1}^{\infty}\frac{e^{-xt}}{t}dt$ from subsection \ref{subsubsec: def01}:
	
	\begin{align}
		\int_{0}^{d}e^{ax}E_1\left(bx\right)dx&=\int_{0}^{d}e^{ax}\left(\int_{1}^{\infty}\frac{e^{-bxt}}{t}dt\right)dx\\
		&=\int_{1}^{\infty}\frac{dt}{t}\int_{0}^{d}dxe^{\left(a-bt\right)x}\\
		&=\int_{1}^{\infty}\frac{dt}{t}\left(\frac{1}{a-bt}e^{\left(a-bt\right)d}-\frac{1}{a-bt}\right)\assume[a<b]\\
		&=\int_{1}^{\infty}\frac{dt}{a}\frac{a}{t\left(a-bt\right)}e^{\left(a-bt\right)d}-\int_{1}^{\infty}dt\frac{1}{t\left(a-bt\right)}\\
		&=\int_{1}^{\infty}\frac{dt}{a}\frac{a-bt+bt}{t\left(a-bt\right)}e^{\left(a-bt\right)d}-\int_{1}^{\infty}dt\frac{1}{t\left(a-bt\right)}\\
		&=\int_{1}^{\infty}\frac{dt}{a}\left(\frac{1}{t}+\frac{b}{a-bt}\right)e^{\left(a-bt\right)d}-\int_{1}^{\infty}dt\frac{1}{t\left(a-bt\right)}\\
		&=\frac{e^{ad}}{a}\int_{1}^{\infty}dt\frac{e^{-dbt}}{t}+\frac{1}{a}\int_{1}^{\infty}dt\frac{be^{\left(a-bt\right)d}}{a-bt}-\int_{1}^{\infty}dt\frac{1}{t\left(a-bt\right)}
	\end{align}
	
	Looking at the first term in the last equation we see the integral representation of $E_1\left(bd\right)$. To simplify to second summand we first apply the substitution $-\left(a-bt\right)=y$ and after applying $\frac{y}{b-a}=x$ see the integral representation of $E_1\left[d\left(b-a\right)\right]$. And finally for the last term in the equation we use the substitution $\frac{1}{a-bt}=z$ and then apply $1-az=w$:
	
	\begin{align}
		\int_{0}^{d}e^{ax}E_1\left(bx\right)dx&=\frac{e^{ad}}{a}E_1\left[bd\right]+\frac{1}{a}\int_{b-a}^{\infty}dy\frac{e^{-yd}}{-y}-\int_{\left(a-b\right)^{-1}}^{0}\frac{b}{a-\frac{1}{z}}\frac{z}{b}\frac{dz}{z^2}\\
		&=\frac{e^{ad}}{a}E_1\left[bd\right]-\frac{1}{a}\int_{1}^{\infty}dx\frac{e^{-d\left(b-a\right)x}}{\left(b-a\right)x}\left(b-a\right)-\int_{1}^{1-\frac{a}{a-b}}\frac{1}{w}\left(-\frac{dw}{a}\right)\\
		&=\frac{e^{ad}}{a}E_1\left[bd\right]-\frac{1}{a}E_1\left[\left(b-a\right)d\right]+\frac{1}{a}\ln{\left[1-\frac{a}{a-b}\right]}\\
		&=\frac{e^{ad}}{a}E_1\left[bd\right]-\frac{1}{a}E_1\left[\left(-a+b\right)d\right]+\frac{1}{a}\ln{\left[\frac{a-b-a}{a-b}\right]}\\
		&=\frac{1}{a}\left\{e^{ad}E_1\left[bd\right]-E_1\left[\left(-a+b\right)d\right]-\ln{\left[-\frac{a-b}{b}\right]}\right\}\\
		&=\frac{1}{a}\left(e^{\left(a-b\right)d}\left\{D\left[bd\right]-D\left[\left(-a+b\right)d\right]\right\}-\ln{\left[1-\frac{a}{b}\right]}\right)\label{eq: generalIntegralOfE4}
	\end{align}
	
	We have to apply the following restrictions:
	
	\begin{subequations}
		\begin{align}
			b&>0\label{eq: restrictions01}\\
			b&>a\label{eq: restrictions02}\\
			bd&>0\overset{\eqref{eq: restrictions02}}{\Rightarrow}d>0
		\end{align}
	\end{subequations}
	
	Equation \eqref{eq: generalIntegralOfE4} is equation (A2b) from the appendix of \cite{boer1990calc}. It is the special case $Y\left(a,b,0,d\right)\coloneqq{}W\left(a,b,d\right)$. It can also be found as equation (R11) and (R22) of \cite{sherman}.
	
	\clearpage
	\printbibliography
	\clearpage
	\appendix
	
	\section{Unused stuff}
	
		\subsection[Definition of Logarithmic Integral li(x)]{Definition of Logarithmic Integral $li\left(x\right)$}
		\label{appsubsec: li}
		
			\begin{gather}
			li\left(x\right)=\int_{0}^{x}\frac{1}{\ln{t}}dt\assume[x>0]\label{eq: li}
			\end{gather}
			
			Note that for $x=1$ the integral diverges (because of the singularity of the integrand $\frac{1}{\ln{t}}$) and for $x>1$ the Cauchy principle value has to be employed to interpret the integral. From the definition of the exponential integral $E_1\left(x\right)$ in equation \eqref{eq: E1} one can see the following relation, when utilizing the substitution $t=e^{-w}$:
			
			\begin{align}
			li\left(e^{-x}\right)&=\int_{0}^{e^{-x}}\frac{1}{\ln{t}}dt\\
			&=\int_{\infty}^{x}\left(-\frac{1}{w}\right)\left(-e^{-w}dw\right)\\
			&=-\int_{x}^{\infty}\frac{e^{-w}}{w}dw\\
			&=-E_1\left(x\right)
			\end{align}
		
		\subsection[Definition of Gamma function]{Definition of Gamma function $\Gamma\left(x\right)$}
		\label{appsubsec: Gamma}
		
			\begin{gather}
			\Gamma\left(x\right)=\int_{0}^{\infty}t^{x-1}e^{-t}dt\assume[x>0]\label{eq: Gamma}
			\end{gather}
		
		\subsection{More general derivation of section \ref{subsubsec: generalIntegralOfE3}}
		\label{appsec: moreGeneralDerivation1}
		
			This more general derivation of subsection \ref{subsubsec: generalIntegralOfE3} is based on that of \autocite[73\psq]{schloemilch}. Start with the integral representation $E_1\left(bx\right)=e^{-bx}\int_{0}^{\infty}\frac{e^{-bxt}}{t+1}dt$ from subsection \ref{subsubsec: intrep02}, which is only valid for $bx>0$. Then multiply the equation with $x^{\mu-1}e^{ax}dx$, while integrating both sides from $x=0$ to infinity:
			
			\begin{align}
			\int_{0}^{\infty}x^{\mu-1}e^{ax}E_1\left(bx\right)dx&=\int_{0}^{\infty}x^{\mu-1}e^{ax}\left[e^{-bx}\int_{0}^{\infty}\frac{e^{-bxt}}{t+1}dt\right]dx\assume[b>0]\\
			&=\int_{0}^{\infty}\frac{dt}{1+t}\int_{0}^{\infty}x^{\mu-1}e^{-\left(-a+b+bt\right)x}dx
			\end{align}
			
			In the last step we just changed the integration order. Now we take a look at the integral over $x$. Using the substitution $kx=u$, and also taking advantage of the abbreviation $k=-a+b+bt$, we get:
			
			\begin{align}
			\int_{0}^{\infty}x^{\mu-1}e^{-kx}dx&=\int_{0}^{\infty}\left(\frac{u}{k}\right)^{\mu-1}e^{-u}\left(\frac{1}{k}du\right)\assume[k>0]\\
			&=k^{-\mu}\int_{0}^{\infty}u^{\mu-1}e^{-u}du\\
			&=\frac{\Gamma\left(\mu\right)}{k^{\mu}}\assume[k>0, \mu>0]
			\end{align}
			
			Here we used the definition of the Gamma function $\Gamma\left(\mu\right)$ from equation \eqref{eq: Gamma}, which is only valid for $\mu>0$. When applying the requirement $k>0$ to the abbreviation $k=-a+b+bt>0$, we can see this yields $-a+b>0$ for the integration limit $t=0$, or equivalently $a<b$. On this result we want to use the substitution $\frac{1}{1+t}=z$, i.e. $t=\frac{1}{z}-1$:
			
			\begin{align}
			\int_{0}^{\infty}x^{\mu-1}e^{ax}E_1\left(bx\right)dx&=\Gamma\left(\mu\right)\int_{0}^{\infty}\frac{1}{\left(1+t\right)}\frac{1}{\left(-a+b\left(1+t\right)\right)^\mu}dt\\
			&=\Gamma\left(\mu\right)\int_{1}^{0}z\frac{1}{\left(-a+\frac{b}{z}\right)^{\mu}}\left(-\frac{1}{z^2}dz\right)\\
			&=\Gamma\left(\mu\right)\int_{0}^{1}\frac{z^{-1}}{\left(\frac{b}{z}-a\right)^{\mu}}\frac{z^{\mu}}{z^{\mu}}dz\\
			&=\Gamma\left(\mu\right)\int_{0}^{1}\frac{z^{\mu-1}}{\left(b-az\right)^{\mu}}dz\assume[a<b, \mu>0]
			\end{align}
			
			Now, we can just set $\mu=1$, use $\Gamma\left(1\right)=1$ and employ the substitution $b-az=x$ to get the desired end result of equation \eqref{eq: generalIntegralOfE3}:
			
			\begin{align}
			\int_{0}^{\infty}e^{ax}E_1\left(bx\right)&=\Gamma\left(1\right)\int_{0}^{1}\frac{1}{b-az}dz\\
			&=\int_{b}^{b-a}\frac{1}{x}\left(-\frac{1}{a}dx\right)\\
			&=-\frac{1}{a}\left[\ln{x}\right]^{b-a}_{x=b}\\
			&=-\frac{1}{a}\left[\ln{\left(b-a\right)}-\ln{\left(b\right)}\right]\\
			&=-\frac{1}{a}\ln{\left(1-\frac{a}{b}\right)}\assume[a<b, 0<b]
			\end{align}
\end{document}